\chapter{Introduction} \label{ch:01}

% MARK: Subject
% Brief, Catchy, Introduction to the subject: Data reductions and software
Data reductions and, by extension, the software enabling them, are an often overlooked aspect of astrophysical research.
They are the foundation upon which scientific results are built.
With ever increasingly complex and sensitive observational techniques and instruments, coupled with the ever-increasing volume of data, the need for efficient and accurate data reductions are a critical aspect of astronomical research.

% More focused subject: SALT and Spectropolarimetry
One such instrument, installed on the \gls{SALT} \citep{SALT_design}, is the \gls{RSS} \citep{SALT_optical_design}.
The \gls{SALT}/\gls{RSS} is currently capable of performing, point-source and diffuse (spatial), long-slit linear spectropolarimetry, allowing for the simultaneous observation of, wavelength dependent, intensity and polarization \citep{SALT_hires}.

% More focused subject: polsalt
The \polsalt\ software is a Python~$2$ based software package that provides a complete reduction pipeline for \gls{SALT} spectro\-polarimetric data, from pre- to final reductions, and the plotting of the spectro\-polarimetric results \citep{polsalt}.
As of 2024, data reductions for \gls{SALT}/\gls{RSS} spectropolarimetric observations are generally completed using an adaption of the beta version of the \polsalt\ software.%
\footnote{As of writing, the versions used for \polsalt\ are as follows: \polsalt~v0.2.dev144, PySALT~v0.5dev, PyRAF~v1.8.1, and \glsxtrshort{IRAF}~v2.14\,.}
% Note \polsalt\ also implements versioning for each method following a `\textit{YYYYMMDD}' format.
A \gls{GUI} is implemented through the beta version, providing limited interactivity during the reduction process.
A user can complete the full reduction process purely using \polsalt, but this does not mean the software is without its limitations.

% MARK: Problem
% Expand on what the problem is: Wavelength calibrations for polsalt
Erroneous inputs or key presses from the user can lead to unexpected program crashes, and the wavelength calibration process is time-consuming and inflexible.
This makes recalibrating the wavelength solutions unfeasible for anything larger than a handful of observations.
Furthermore, the wavelength calibration of the $O$- and $E$-beams are entirely independent of each other.
This allows good fits to each polarimetric beam, but does not ensure that the wavelength solutions are consistent across the two beams, leading to potential systematic errors in the final results.

% MARK: Problem Statement
% Problem Statement and Outline of Aims: Development of STOPS
To address these challenges, the objective of this study is to develop supplementary tools to aid in the spectro\-polarimetric reduction process.
The aims of these supplementary tools, named \stops, are:
\begin{itemize}
    \item to provide a more interactive approach to the wavelength calibration process,
    \item to allow integration of alternate wavelength calibration methods into the standard \polsalt\ reduction procedure,
    \item to ensure the accuracy of spectropolarimetric wavelength solutions, and
    \item to improve the overall efficiency and wavelength calibration process for spectro\-polarimetric wavelength calibrations.
\end{itemize}
The \stops\ software should allow for efficient and accurate wavelength calibrations.
It is designed to be used in conjunction with the \polsalt\ software, allowing for more interactive wavelength calibration processes to be used in place of the built-in \polsalt\ \texttt{wavelength calibration} method.
This means that \stops\ does not perform the wavelength calibration itself, but should rather parse the \gls{FITS} files from and to the \polsalt\ format before and after the wavelength calibration process, respectively.
\stops\ should also be expected to handle any additional calibrations which are completed during the \polsalt\ wavelength calibration process, such as the wollaston tilt corrections and \gls{CRR}.

% MARK: Scope
% Scope and objectives: What \stops\ can and can't do
Further secondary aims were to make the installation and usage of the software as user-friendly as possible, and to ensure the software is easily maintainable and extendable.
These secondary aims, thankfully, relate to design choices, and are controlled through the choice of programming language and distribution method.
Python~$3$ is the language of choice due to the current widespread use of Python in the astronomical community.
The software will be distributed through GitHub for version control, with the option to later publish to the \gls{PyPI} once a stable release is developed.
Lastly, Python relies on the Python package manager, pip, which allows the installation of \stops\ to be handled natively.

% MARK: Importance
The development of \stops\ was initiated to be used as part of ongoing research which uses \gls{SALT} to study the polarimetry properties of \glspl{AGN}, in particular, blazars.
\Glspl{AGN} are the cores of galaxies powered by accretion onto a super-massive black hole.
Blazars represent the subclass of radio-loud \glspl{AGN} with relativistic jets closely aligned to our line of sight ($\theta \lesssim 10$\degree), known for their rapid and high degree of variability across the electromagnetic spectrum \citep{Urry_1995}.
% MARK: Blazar SEDs
The \gls{SED} of blazars shows two clear components:
a lower energy component produced through leptonic synchrotron processes (spanning the radio to \gls{UV}/soft X-ray regimes)
and a higher energy component (spanning the X-ray to $\gamma$-ray regimes) which is proposed to be produced through either leptonic or hadronic processes \citep{Bottcher_2013}.
%
% MARK: Blazar specpol
The optical emission from blazars comprises of polarized, non-thermal, synchrotron emission arising in the jets, with underlying non-polarized, thermal, emission arising from the host galaxy, dusty torus, and accretion disk components  \citep{Ghisellini_2009}.
Optical spectropolarimetry, coupled with multi-wavelength observations, during both flaring and quiescent states can be used to disentangle the polarized and non-polarized components in the blazar's \gls{SED}, providing better constraints for the non-thermal particle distribution \citep{Schutte_COSPAR, Schutte4C0102}.

The \gls{SALT}/\gls{RSS} PG$0300$ grating, and accompanying \gls{Ar} arc lamp, provided the widest wavelength range and highest throughput, making them ideal for optical observations of blazars during differing states of flaring and quiescence.
These spectro\-polarimetric observations are, however, particularly challenging to reduce as the \gls{Ar} arc lamp exhibits sparse spectral features across the wavelength range, with a partial overlap of a higher order at longer wavelengths.
Due to these difficulties, a backlog of unanalyzed data existed.



% MARK: Significance
Integrating alternate wavelength calibrations for \gls{SALT}/\gls{RSS} spectro\-polarimetric observations has broad implications for astronomical research, particularly in studies involving sources which display high degrees of polarization, such as high-energy sources.
By providing alternate reliable and efficient methods of wavelength calibration, this dissertation directly enhances the efficiency of the \gls{SALT}/\gls{RSS} to produce high-quality spectropolarimetric results, allowing the interplay between polarization and spectral features to be further investigated.
% The development of tools to streamline and enhance the wavelength calibration process is crucial for maximizing the scientific output of SALT spectropolarimetry. The \stops\ pipeline has been designed to address these challenges, providing an efficient and accurate approach to reducing spectropolarimetric data.

\section{Outline}

\noindent The layout of the rest of the dissertation is as follows:

\autoref{ch:02} lays a foundation for spectroscopy, polarimetry, and spectropolarimetry as well as the implementation of these principles through instrumentation, focusing specifically on principles as relating to the \gls{SALT} \gls{RSS}.
These principles provide an understanding of spectropolarimetric data as well as describe the reduction and calibration processes to be completed for the acquisition of spectropolarimetric results.

\autoref{ch:03} describes the existing \polsalt\ (used for spectropolarimetric reductions) and \iraf\ (used for wavelength calibrations) software, the developed \stops\ software, and provides a general reduction process for spectropolarimetric data.
The principles, application, and challenges faced when using the existing software for spectropolarimetric data reductions are broadly described, with greater emphasis placed on the developed software, \stops, which was designed to streamline the data reduction process and overcome the limitations of the existing software.

\autoref{ch:04} provides the testing of the developed software and discusses its application within published articles and proceedings.
Testing was conducted on a `per-module' basis aligning with the usage of \stops.

\autoref{ch:05} concludes the dissertation body by summarizing the development and testing of the \stops\ software, noting the current application of the software in publications, and describing possible routes for future development, focusing on stability, further integration with \polsalt, and development of improved features.

Finally, the appendices, \autoref{app:reduction},~\ref{app:code},~and~\ref{app:papers}, contain a working reduction procedure (referred to in \autoref{ch:03}), the \stops\ source code, and Proceedings produced from conferences (referred to in \autoref{ch:04}), respectively.
