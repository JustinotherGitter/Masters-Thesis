\chapter{Introduction}

% 1 Section summary of research. More complete abstract without results.
\todo{
    Very short intro to Spectroscopy, Polarization, and Spectropolarimetry and their importance in astronomy
}

\todo{
    Focus on AGN implications and implementations such as the types of objects and a short history for each type of object, Blazar focus with specification on BL Lacs and FSRQs, the Unified Model, \sout{The Blazar sequence}.\\\textit{Brian's comment:} Highlight importance of polarimetry for understanding emission and how that plays a role in AGN.
}

\todo{
    Basics of modelling (Different energy/wavelength ranges used and what the models tell us about emission processes/structure) so that Hester's results can be noted for applications of the pipeline.
}

\section{Outline}
\todo{Problem Statement, \textbf{VERY IMPORTANT}, roughly a sentence but problem thoroughly fleshed out.}
\todo{General layout of Dissertation}

% MARK: Problem Statement



% MARK: Outline ch02
% The layout of the rest of the thesis is as follows.
% \autoref{ch:02} lays a deeper foundation for spectroscopy, polarimetry, and spectropolarimetry such that these principles may be understood and applied through software to spectropolarimetric reductions.
% It further describes the \gls{SALT} and the \gls{RSS} and their use in spectropolarimetric observations.
% MARK: Outline ch03
% \autoref{ch:03} describes the software tools developed to streamline the data reduction process.
% MARK: Outline ch04
% \autoref{ch:04} demonstrates the application of these tools to spectropolarimetric observations of the blazar 3C~279.
% MARK: Outline ch05
% \autoref{ch:05} presents the results of the pipeline and the implications of the results.
% MARK: Outline Appendix
% Finally, \autoref{ch:Appendix} contains the code used in the pipeline and the data reduction process.


% HEASA 2021
% Active Galactic Nuclei (AGN) are the centers of galaxies powered by accretion onto a super-massive black hole. The accretion can power relativistic jets which produce non-thermal emission over multiple wavelengths \cite{1995PASP..107..803U}. Blazars are a subclass of AGN where the direction of the jet lies very close to our line of sight. The highly Doppler boosted emission from the jet results in high apparent luminosities.

% At optical wavelengths the observed emission is a superposition of polarized non-thermal synchrotron emission, arising from the jet, and non-polarized thermal emission, arising from the disc, broad line region, torus, and host galaxy \cite{galaxies5030052}.

% The Robert Stobie Spectrograph (RSS) on the Southern African Large Telescope (SALT) \cite{burgh2003prime} \cite{kobulnicky2003} is capable of long-slit spectropolarimetry and has been used to observe blazars in flaring and \edit{quiescent} states. Data reduction of these observations is processed using the SALT spectropolarimetry reduction package {\sc polsalt}.\footnote{https://github.com/saltastro/polsalt} {\sc polsalt} allows for the full reduction, from pre-reduction to the extraction of the target spectra, and the measuring of the polarization. However, it does not allow for much flexibility with the wavelength calibrations, nor does it provide tools to confirm that the \edit{wavelength calibrations lead to similar wavelength calibrated spectra for the O and E beams}. %The wavelength solutions can be parsed out from the FITS files but this can become tedious for a larger amount of observations, or for multiple sources.

% In order to streamline the data reduction we are developing additional tools to work in conjunction with {\sc polsalt}. We present the overview of this pipeline and demonstrate its application to an observation of 3C 279 taken in 2017.

% HEASA 2022
% Blazars are a radio-loud sub-set of Active Galactic Nuclei (AGN), where the jet is aligned very closely to our line of sight and whose observed emission varies over time scales from hours to decades \citep{Urry}. The optical emission observed from blazars is often dominated by the polarized, non-thermal emission arising in the jets, but there is also underlying non-polarized, thermal emission arising from the host galaxy, dusty torus, and accretion disk components \citep{Ghisellini}. When a blazar varies between flaring and quiescent states, the relative strength of the thermal component to the non-thermal component changes.

% The Spectral Energy Distribution (SED) of blazars shows two clear components: a lower energy component produced through leptonic synchrotron processes (covering the radio to the UV/soft X-ray regimes) and a higher energy component (covering the X-ray to the $\gamma$-ray regimes) which can be produced through either leptonic or hadronic processes \citep{Bottcher_2013}. Optical spectropolarimetry observations, coupled with multi-wavelength observations during both flaring and quiescent states, can be used to disentangle the polarized and non-polarized components in the blazar's SED, providing better constraints for the non-thermal particle distribution \citep{Schutte_COSPAR, Schutte_2022}.

% To this end, spectropolarimetric observations of blazars during different states of activity were taken using SALT \citep{SALT} and the RSS \citep{RSS}. RSS spectropolarimetry data is reduced with the dedicated \textsc{polsalt} pipeline \citep{polsalt}. In order to facilitate the blazar observations we have developed supplementary tools to provide a more interactive approach to the wavelength calibration, allowing for improved accuracy for the O \& E beam wavelength solutions \citep{Cooper_HEASA}. Furthermore, spectropolarimetric standards, which were also observed with SALT and the RSS, were reduced alongside the blazar observations to test that the developed tools correctly performed the wavelength calibration. Here we present an overview of the pipeline as well as the preliminary results of a spectropolarimetric standard, namely Hiltner~652, and for the blazar 3C~279.
