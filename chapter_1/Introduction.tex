\chapter{Introduction} \label{ch:01}

% MARK: Subject
% Brief, Catchy, Introduction to the subject: Data reductions and software
Data reductions, and by extension the software enabling them, are an often overlooked aspect of astrophysical research.
They are the foundation upon which scientific results are built.
With ever increasingly complex and sensitive observation techniques and instruments, coupled with the ever-increasing volume of data, the need for efficient and accurate data reductions are a critical aspect of astronomical research.

% More focused subject: SALT and Spectropolarimetry
One such instrument, installed on the \gls{SALT} \citep{SALT_design}, is the \gls{RSS} \citep{SALT_optical_design}.
The \gls{SALT}/\gls{RSS} is capable of performing long-slit spectropolarimetry \citep{SALT_hires}, allowing for the simultaneous observation of, wavelength dependent, intensity and polarization.
The \polsalt\ software is a Python~$2$ based software package that provides a complete reduction pipeline for \gls{SALT} spectro\-polarimetric data, from pre- to final reductions, and the plotting of the spectro\-polarimetric results \citep{polsalt}.
As of 2024, data reductions for \gls{SALT}/\gls{RSS} spectropolarimetric observations are generally completed using an adaption of the beta version of the \polsalt\ software.%
% \footnote{As of writing, the versions used for \polsalt\ are as follows: \polsalt\ v0.2.dev144, pysalt v0.5dev, \iraf v2.14, and pyraf v1.8.1}

% MARK: Problem
% Expand on what the problem is: Wavelength calibrations for polsalt
A user can complete the full reduction process purely using \polsalt, with \gls{GUI} interactivity made possible by the beta version, but this does not mean the software is without its limitations.
Erroneous inputs or key presses from the user can lead to unexpected program crashes, and the wavelength calibration process is time-consuming and inflexible.
This makes recalibrating the wavelength solutions unfeasible for anything larger than a handful of observations.
Spectro\-polarimetric observations performed with the \gls{SALT}/\gls{RSS} using the PG$0300$ grating are particularly challenging to reduce due to the \gls{SALT} \gls{Ar} arc lamp.
The \gls{Ar} arc lamp exhibits sparse spectral features across the wavelength range, with a partial overlap of a higher order at longer wavelengths.

% MARK: Problem Statement
% Problem Statement and Outline of Aims: Development of STOPS
To address these challenges, supplementary tools are developed to aid in the spectro\-polarimetric reduction process.
The aims of these supplementary tools, named \stops, are:
\begin{itemize}
    \item to provide a more interactive approach to the wavelength calibration process,
    \item to allow integration of alternate wavelength calibration methods into the standard \polsalt\ reduction procedure,
    \item to ensure the accuracy of spectropolarimetric wavelength solutions, and
    \item to improve the overall efficiency and wavelength calibration process for spectro\-polarimetric wavelength calibrations.
\end{itemize}

% MARK: Scope
% Scope and objectives: What \stops\ can and can't do
The \stops\ software should allow for efficient and accurate wavelength calibrations.
It is designed to be used in conjunction with the \polsalt\ software, allowing for more interactive wavelength calibration processes to be used in place of the built-in \polsalt\ \texttt{wavelength calibration} method.
This means that \stops\ does not perform the wavelength calibration itself, but should rather parse the \gls{FITS} files from and to the \polsalt\ format before and after the wavelength calibration process, respectively.
\stops\ should also be expected to handle any additional calibrations which are completed during the \polsalt\ wavelength calibration process, such as wollaston tilt corrections and \gls{CRR}.

% MARK: Importance
% Explain importance: need for accurate and auditable wavelength solutions
The \gls{SALT}/\gls{RSS} PG$0300$ grating, and accompanying \gls{Ar} arc lamp, provided the widest wavelength range and highest throughput, making it ideal for optical observations of blazars during differing states of flaring and quiescence.
Due to the difficulties mentioned above, a backlog of unanalyzed data existed.
%%% Explain why solving this problem is significant and how it advances the field.
%%% Discuss the personal or academic motivation behind the project.
%% a need to compare wavelength solutions across the perpendicular O and E polarization beams.

%%% This allows for the study of the polarization properties of blazars within these wavelength regimes, which are often dominated by a polarized, non-thermal emission component arising in the jets, with an underlying non-polarized, thermal emission component arising from the host galaxy, dusty torus, and accretion disk components.

% C. The method of procedure and treatment of findings.
% 3. State the purpose of the paper (how subject is handled). What is the point of view and emphasis of the paper? How does the purpose differ from that of other papers on the same subject?

%%%   * Mention Python's integration and the role of PyPI for disseminating new tools like STOPS.
%%%   * How STOPS fits into the reduction workflow

% The development of tools to streamline and enhance the wavelength calibration process is crucial for maximizing the scientific output of SALT spectropolarimetry. The \stops\ pipeline has been designed to address these challenges, providing an efficient and accurate approach to reducing spectropolarimetric data.

% Spectropolarimetric observations are invaluable in astrophysics as they offer a means to measure the polarization of light, which can reveal crucial information about the magnetic fields, emission mechanisms, and geometries of distant objects. In particular, SALT/RSS spectropolarimetry has proven to be highly effective for the study of high-energy astrophysical sources such as blazars and gamma-ray bursts (GRBs).

% MARK: Blazars / SED
\gls{AGN} are the cores of galaxies powered by accretion onto a super-massive black hole.
Blazars represent the subclass of radio-loud \gls{AGN} with relativistic jets closely aligned to our line of sight ($\theta \lesssim 10$\degree), known for their rapid and high degree of variability across the electromagnetic spectrum \citep{Urry_1995}.
Polarized, non-thermal, synchrotron emission arising in the jets, with underlying non-polarized, thermal, emission arising from the host galaxy, dusty torus, and accretion disk components comprises the optical emission observed from blazars \citep{Ghisellini_2009}.
When a blazar varies between quiescent and flaring states, the relative strength of the non-thermal component to the thermal component changes, presenting unique opportunities to disentangle the underlying emission mechanisms.

% The \gls{SED} of blazars shows two clear components: a lower energy component produced through leptonic synchrotron processes (covering the radio to \gls{UV}/soft X-ray regimes) and a higher energy component (covering the X-ray to $\gamma$-ray regimes) which can be produced through either leptonic or hadronic processes \citep{Bottcher_2013}.
% Optical spectropolarimetry observations, coupled with multi-wavelength observations during both flaring and quiescent states, can be used to disentangle the polarized and non-polarized components in the blazar's \gls{SED}, providing better constraints for the non-thermal particle distribution \citep{Schutte_COSPAR, Schutte_2022}.
% These observations play a pivotal role in constraining theoretical models where understanding the polarization of light can lead to a deeper comprehension of jet dynamics, particle acceleration, and the role of magnetic fields in energy transport.

% Blazars and GRBs are known for their intense emission across the electromagnetic spectrum. These sources often exhibit highly polarized light, which is directly linked to synchrotron emission in relativistic jets. Through spectropolarimetric observations, it is possible to investigate the structure and evolution of these jets, providing insights into their magnetic field configurations and energy dissipation mechanisms.

% MARK: Significance
Integrating alternate wavelength calibrations for \gls{SALT}/\gls{RSS} spectro\-polarimetric observations has broad implications for astronomical research, particularly in studies involving sources which display high degrees of polarization, such as high-energy sources.
By providing alternate reliable and efficient methods of wavelength calibration, this thesis directly enhances the capability of the \gls{SALT}/\gls{RSS} to produce high-quality spectropolarimetric results, allowing the interplay between polarization and spectral features to be further investigated.

\section{Outline}

\noindent The layout of the rest of the thesis is as follows:

\autoref{ch:02} lays a foundation for spectroscopy, polarimetry, and spectropolarimetry as well as the implementation of these principles through instrumentation, focusing specifically on principles as relating to the \gls{SALT} \gls{RSS}.
These principles provide an understanding of spectropolarimetric data as well as describe the reduction and calibration processes to be completed for the acquisition of spectropolarimetric results.

\autoref{ch:03} describes the existing \polsalt\ (used for spectropolarimetric reductions) and \iraf\ (used for wavelength calibrations) software, the developed \stops\ software, and provides a general reduction process for spectropolarimetric data.
The principles, application, and challenges faced when using the existing software for spectropolarimetric data reductions are broadly described, with greater emphasis placed on the developed software, \stops, which was designed to streamline the data reduction process and overcome the limitations of the existing software.

\autoref{ch:04} provides the testing of the developed software and discusses its application within published articles and proceedings.
Testing was conducted on a `per-module' basis aligning with the usage of \stops.

\autoref{ch:05} concludes the thesis body by summarizing the development and testing of the \stops\ software, noting the current application of the software in publications, and describing possible routes for future development, focusing on stability, further integration with \polsalt, and development of improved features.

Finally, the appendices, \autoref{app:reduction},~\ref{app:code},~and~\ref{app:papers}, contain a working reduction procedure (referred to in \autoref{ch:03}), the \stops\ source code, and Proceedings produced from conferences (referred to in \autoref{ch:04}), respectively.
