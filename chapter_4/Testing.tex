% MARK: Testing
\chapter{Testing}

This chapter contains an overview of the testing performed for and during the development of \stops. The tests are divided into two main categories: \stops\ software tests (\autoref{sec:test_stops}) and reduction specific tests (\autoref{sec:test_reduction}). Software tests cover the methods implemented and error handling in \stops, specifically for \texttt{split} (\autoref{subsec:test_split}), \texttt{join} (\autoref{subsec:test_join}), \texttt{correlate} and \texttt{skylines} (\autoref{subsec:test_corr_sky}), while reduction tests cover verifying the alternate wavelength solutions (\autoref{subsec:test_wavelength}) and polarization parameters (\autoref{subsec:test_polarization}).

% MARK: Software tests
\section{\stops\ Development} \label{sec:test_stops}

As development is an iterative process, only the major tests that were performed are discussed. The tests were, predominantly, to ensure that the input to \stops\ is parsed correctly and that the output of the relevant \stops\ methods align with the expected input for the relevant \iraf\ or \polsalt\ tasks.

% Why the pipeline is better
The rigorous error handling in \stops\ ensures that the user is informed of any issues that arise during the reduction process. This is particularly important as \stops\ was developed to enable a faster reduction process compared to that of pure \iraf\ or \polsalt.

% Which tools were used/replaced/improved


% MARK: Split tests
\subsection{Testing the \stops\ \texttt{split} method} \label{subsec:test_split}

\todo{Tests performed to match \stops\ output to that of \iraf\.}

% MARK: Join tests
\subsection{Testing the \stops\ \texttt{join} method} \label{subsec:test_join}

\todo{Tests performed to match \stops\ output to that of \polsalt\.}
\todo{JOIN uses direct function of \polsalt, specifically the wollaston corrections, but updated to python3.}

% MARK: Cross Correlate/Skylines tests
\subsection{Testing the \stops\ \texttt{correlate} and \texttt{skylines} methods} \label{subsec:test_corr_sky}

\todo{}

% MARK: Reduction tests
\section{Reduction specific tests} \label{sec:test_reduction}

\todo{Reduction specific tests refer to tests performed to ensure no errant effects are introduced during the reduction process...}

\todo{
    \begin{itemize}
        \item Brief discussion of sources used as part of testing.
        \item General discussion of testing (I.E. not this test was done specifically this source, more along the lines of these tests were done to check this issue, seen here using this source for example.)
    \end{itemize}
}

% MARK: Wavelength tests
\subsection{Wavelength Solutions} \label{subsec:test_wavelength}

\todo{Wavelength solution tests
    \begin{itemize}
        \item Full frame wavelength solution checks
        \item O/E correlation checks
        \item RMS comparisons of polsalt to stops
    \end{itemize}
}

% MARK: Polarization tests
\subsection{Polarization Parameters} \label{subsec:test_polarization}

% \todo{}

\todo{Add all tests done and comparisons.
    \begin{itemize}
        \item 3C 279
        \item 4C+01.02
        \item David data (not in next section publications because still during pipeline development. Reductions done through polsalt, but after publication used as preliminary testing data)
    \end{itemize}
}

% MARK: Spectropolarimetric Standards
\subsubsection{Spectropolarimetric Standards}

\todo{Testing Spectropolarimetric standards (4 highly polarized, 2 non-polarized)
    \begin{itemize}
        \item (Bulleted list same as ch05, but without the science results) 
        \item Background on objects
        \item Reductions
        \item Actual results - comparison of polsalt results to supplementary pipeline results
    \end{itemize}
}
