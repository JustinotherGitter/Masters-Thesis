\begin{table}[t]

    \centering

    \caption{Properties of the identified absorption lines within the spectra. The measured redshifts of the identified features correspond neatly to two distinct redshifts. Table adapted from \citet{Buckley191221B}.}
    \label{table:grb_lines}

    \begin{tabular}{lccccc}
        \toprule
        Line ID &
        \begin{tabular}[c]{@{}c@{}}Rest\\Wavelength (\AA)\end{tabular} &
        \begin{tabular}[c]{@{}c@{}}Observed\\Wavelength (\AA)\end{tabular} &
        \begin{tabular}[c]{@{}c@{}}\gls{FWHM}\\(\AA)\end{tabular} &
        \begin{tabular}[c]{@{}c@{}}\glsxtrshort{EW}\\(\AA)\end{tabular} &
        $z$ \\
        \midrule
        Fe~II & $2343$ & $5032.35 \pm 1.44$ & $5.44 \pm 1.43$ & $0.83 \pm 0.26$ & $1.148$ \\
        Fe~II & $2599$ & $5096.67 \pm 2.19$ & $4.70 \pm 2.26$ & $0.45 \pm 0.23$ & $0.961$ \\
        Fe~II & $2382$ & $5114.49 \pm 0.90$ & $4.67 \pm 0.96$ & $1.10 \pm 0.23$ & $1.147$ \\
        Mg~II & $2795$ & $5481.56 \pm 1.57$ & $4.10 \pm 1.67$ & $0.67 \pm 0.21$ & $0.961$ \\
        Mg~II & $2802$ & $5494.59 \pm 2.19$ & $3.93 \pm 2.29$ & $0.45 \pm 0.19$ & $0.961$ \\
        Fe~II & $2599$ & $5552.48 \pm 1.25$ & $4.01 \pm 1.25$ & $0.62 \pm 0.19$ & $1.136$ \\
        Mg~I  & $2852$ & $5581.78 \pm 0.91$ & $4.44 \pm 0.91$ & $1.05 \pm 0.22$ & $0.957$ \\
        Mg~II & $2795$ & $6002.98 \pm 0.37$ & $4.56 \pm 0.40$ & $2.08 \pm 0.25$ & $1.148$ \\
        Mg~II & $2802$ & $6018.71 \pm 0.14$ & $4.38 \pm 0.43$ & $1.83 \pm 0.24$ & $1.148$ \\
        Mg~I  & $2852$ & $6124.64 \pm 1.28$ & $4.10 \pm 1.30$ & $0.69 \pm 0.22$ & $1.128$ \\
        \bottomrule
    \end{tabular}

\end{table}
