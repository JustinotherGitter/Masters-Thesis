\chapter{Testing and Application}

This chapter contains an overview of the testing performed for the development of \stops\ (\autoref{sec:test_stops}) and the checking of the replaced wavelength solutions (\autoref{sec:test_wav}), as well as the application of \stops\ on observations (\autoref{sec:results_unpub}) and its application in publications (\autoref{sec:results_pub}).

% MARK: Test STOPS
\section[Testing \textsc{stops}]{Testing \stops} \label{sec:test_stops}

% No \polsalt\ tests or tests on pre-reduced \gls{FITS} files. (Trusted accurate)
% General discussion of testing (I.E. not this test was done specifically this source, more along the lines of these tests were done to check this issue, seen here using this source for example.)
The main challenge faced when developing \stops\ was ensuring that the software was compatible with both the \polsalt\ and \iraf\ file structures. As development is an iterative process, \stops\ was continually checked to ensure compatibility such that the varying \stops\ method inputs were correctly parsed, and that their outputs were parsable by the relevant \iraf\ tasks or \polsalt\ methods.

To this end, observations which were verified to have been accurately reduced were duplicated for testing purposes, allowing for continual checks of the \stops\ pipeline to be made during the development process. As the \stops\ \texttt{split} and \texttt{join} methods are designed to convert between the \polsalt\ and \iraf\ file structures, greater emphasis was made to ensure that the output of both methods provided accurate and consistent results.

% % Why the pipeline is better
% The rigorous error handling in \stops\ ensures that the user is informed of any issues that arise during the reduction process. This is particularly important as \stops\ was developed to enable a faster reduction process compared to that of pure \iraf\ or \polsalt.

% MARK: STOPS split tests
\subsection{Testing the \texttt{split} Method} \label{subsec:test_split}

The \stops\ \texttt{split} method requires any \polsalt\ pre-reduced (`mxgbp-' prefixed) \gls{FITS} files as input and outputs \iraf\ compatible (`(beam|arc)(O|E)-' prefixed) \gls{FITS} file structures. As no `split' \gls{FITS} files are created during pure \polsalt\ reductions, the \stops\ \texttt{split} method was tested by comparing the pre-reduced \polsalt\ files to the \texttt{split} method's output files, ensuring the correct structure and data integrity of the files handed off to \iraf.

\newcommand\mc[1]{\multicolumn{1}{c|}{#1}} % handy shortcut macro

\begin{table}[t]
    \centering
    \begin{tabular}{ccccccc}
        \hline
        Filename            & No. & Name    & Type       & Cards & Dimensions   & Format  \\ \hline
        \mc{\multirow{4}{*}{\parbox[c]{1.8cm}{mxgbpP\-20170328\-0054.fits}}}
                            & $0$ & \gls{PRIMARY} & PrimaryHDU & $161$   & ()           &  \\
        \mc{}               & $1$ & \gls{SCI}     & ImageHDU   & $19$    & ($3199$, $1028$) & float32 \\
        \mc{}               & $2$ & \gls{VAR}     & ImageHDU   & $8$     & ($3199$, $1028$) & float32 \\
        \mc{}               & $3$ & \gls{BPM}     & ImageHDU   & $8$     & ($3199$, $1028$) & uint8   \\
        \mc{beamo0054.fits} & $0$ & \gls{PRIMARY} & PrimaryHDU & $162$   & ($3199$, $474$)  & float32 \\
        \mc{beame0054.fits} & $0$ & \gls{PRIMARY} & PrimaryHDU & $162$   & ($3199$, $474$)  & float32 \\ \hline
    \end{tabular}
    \caption{A comparison of the \polsalt\ pre-reduced \texttt{mxgbpP201703280054.fits} file, and the \stops\ \texttt{split} \texttt{beamo0054.fits} and \texttt{beame0054.fits} file contents.}
    \label{table:split_info}
\end{table}


\autoref{table:split_info} shows the \gls{FITS} file information for the files before and after splitting. The split \gls{FITS} files contain the split \gls{SCI} extension data and the \gls{PRIMARY} header from the pre-reduced files, with any Header or Data differences mentioned below.

The header is left mostly untouched, and is only updated to represent the new data type and shape:
the `BITPIX' value is updated, from $8$ to $-32$, and the `NAXIS' value is updated, from $0$ to $2$;
the `NAXIS1' and `NAXIS2' keywords are added, and their values are set to the new split \gls{SCI} data shape;
and the `EXTEND' keyword is removed.%
\footnote{The `EXTEND' keyword indicates that the \gls{FITS} file contains multiple extensions while the `NAXIS1' and `NAXIS2' keywords indicate the shape and size of the data stored in the relevant extension.}
This accounts for the discrepancy in the `Cards' between the \polsalt\ and \stops\ file header entries in \autoref{table:split_info}.

The \polsalt\ \gls{SCI} data is unmodified when copying the data to the \stops\ \gls{FITS} file, but only includes half of the data, of the relevant polarization beam, with a cropping which defaults to $40$~pixels (see \autoref{subsec:stops_split}), introduced to the top- and bottom-most rows of the \polsalt\ data. This accounts for the discrepancy in the `Dimensions' between the \polsalt\ and \stops\ files in \autoref{table:split_info}.

This output file structure was chosen for \iraf\ compatibility, and was tested for robustness over multiple grating and articulation angles, as well as with various data sets to ensure that the \texttt{split} method was robust and reliable.

% MARK: STOPS join tests
\subsection{Testing the \texttt{join} Method} \label{subsec:test_join}

The \texttt{join} method requires both an \iraf\ database with wavelength solutions (or a custom wavelength solution) for both polarimetric beams and the \polsalt\ pre-reduced files as input and outputs \polsalt\ \texttt{spectra extraction} compatible (`wmxgbp-' prefixed) \gls{FITS} file structures. Ensuring that the output format was correct was paramount as the \polsalt\ \texttt{spectra extraction} method is unable to process the files otherwise, thus halting the reduction process. Thankfully, the \texttt{join} method output could be compared to the \polsalt\ \texttt{wavelength calibration} method output files, ensuring that any changes introduced by the \stops\ pipeline were well characterized.

\newcommand\mc[1]{\multicolumn{1}{c|}{#1}} % handy shortcut macro

\begin{table}[t]
    \centering
    \begin{tabular}{}
        \hline
        \\ \hline
        \hline
    \end{tabular}
    \caption{}
    \label{table:join_info}
\end{table}


\autoref{table:join_info} shows the \gls{FITS} file information for both the \polsalt\ and \stops\ wavelength calibrated files. Other than the `Dimensions' of each `ImageHDU' extension,%
\footnote{The `Dimensions' differ due to the before mentioned cropping of the top- and bottom-most rows of the data.}
the \gls{FITS} files are identical in structure.

Although the `Cards' count is the same, minor differences across the headers are present. The `HISTORY' keyword, which contains the \polsalt\ `CRCLEAN' parameters and which default to `upper= 4.0, lower= 1.5, sigmaveto= 2.0', is left as `None' in the \stops\ file.%
\footnote{The \polsalt\ pipeline performs cosmic ray cleaning using a $10\sigma$ spike to cull cosmic rays. See the \polsalt\ \protect\href{https://github.com/saltastro/polsalt/blob/master/polsalt/specpolwavmap.py\#L132}{source code} for more information.}
Although \stops\ performs cosmic ray cleaning (see \autoref{subsec:stops_join}), the parameters are not stored in the header as \polsalt\ and \stops\ implement different methods for cosmic ray cleaning. Other minor differences such as the date-times stored in the `SAL-TLM' and `SMOSAIC' keywords may also differ as they contain the date-times relating to the completion of the \polsalt\ pre-reductions. This accounts for the differences in the `Cards' between the \polsalt\ and \stops\ file header entries in \autoref{table:join_info}.

\begin{figure}
    \centering
    \begin{subfigure}[b]{\textwidth}
        \centering
        \includegraphics[width=\textwidth]{4_diff_SCI.pdf}
        \caption{The difference in the \gls{SCI} extensions.}
        \label{subfig:join_SCI}
    \end{subfigure}
    \hfill
    \begin{subfigure}[b]{\textwidth}
        \centering
        \includegraphics[width=\textwidth]{4_diff_VAR.pdf}
        \caption{The difference in the \gls{VAR} extensions.}
        \label{subfig:join_VAR}
    \end{subfigure}
    \hfill
    \begin{subfigure}[b]{\textwidth}
        \centering
        \includegraphics[width=\textwidth]{4_diff_BPM.pdf}
        \caption{The difference in the \gls{BPM} extensions.}
        \label{subfig:join_BPM}
    \end{subfigure}
    \hfill
    \begin{subfigure}[b]{\textwidth}
        \centering
        \includegraphics[width=\textwidth]{4_diff_WAV.pdf}
        \caption{The difference in the \gls{WAV} extensions.}
        \label{subfig:join_WAV}
    \end{subfigure}
    \caption{The difference of the \gls{FITS} file extensions between the \polsalt\ and \stops\ (`wmxgbp-' prefixed) wavelength calibrated files. Figures created using both the \polsalt\ and \stops versions of the \polsalt\ \texttt{spectral extraction} input.}
    \label{fig:join_in_out_diff}
\end{figure}

\autoref{fig:join_in_out_diff} shows the differences in the data between the \polsalt\ and \stops\ wavelength calibrated files. It can be seen in \autoref{subfig:join_VAR} that the \gls{VAR} extensions are identical. The \gls{SCI} extensions differ only in that the cosmic ray cleaning has been applied to the \stops\ data, whereas the \polsalt\ data applies a mask to the cosmic rays using the \gls{BPM} extension. The \gls{BPM} extensions are also masked to account for the valid wavelength calibrated region (\autoref{subfig:join_WAV}). The \gls{WAV} extensions contain the differing wavelength solutions and as such naturally differ. This accounts for the differences in the data between the \polsalt\ and \stops\ files.

Finally, the \stops\ \texttt{join} method was tested to ensure compatibility and correctness of the output data in comparison to \polsalt. This involved testing the \texttt{join} method with various data sets to ensure that the output files were accurate and consistent.

% MARK: Check Wav. Cal.'s
\section{Wavelength Solution Checks} \label{sec:test_wav}

% No \iraf\ tests or tests on \iraf\ \gls{FITS} files. (Trusted accurate)
The secondary challenge encountered when developing \stops\ was ensuring that the wavelength solutions parsed by \stops\ were unaffected by the pipeline and that they were consistent with those created by \polsalt. This was achieved through the \texttt{correlate} and \texttt{skylines} methods, which were designed to validate the wavelength solutions produced by \iraf.

Before the \polsalt\ wavelength calibrations were replaced with the \iraf\ wavelength calibrations, the accuracy of the new wavelength solutions needed to be validated. This was done both through the \iraf\ tasks, ensuring an accurate wavelength solution, and through the \stops\ \texttt{correlate} and \texttt{skylines} methods, allowing the integration of the wavelength solutions to be validated.
% \gls{RMS} results returned from each

% Wavelength solution validation from correlate and skylines results.
%     \begin{itemize}
%         \item Figures showing Correlate and Skyline results.
%         \item RMS comparisons between \polsalt\ and \iraf\ to quantify differences.
%         \item Any Figures for RMS comparisons or wavelength validation?
%     \end{itemize}

% Testing \iraf\ wavelength solution
% \begin{itemize}
%     \item The \texttt{skylines} and \texttt{correlate} outputs are tests of the wavelength calibration.
% \end{itemize}

% Wavelength solution tests were conducted to validate the accuracy of the \stops\ wavelength calibration process. This included:
% \begin{itemize}
%     \item Full frame wavelength solution checks to ensure comprehensive calibration.
%     \item O/E correlation checks to verify the consistency of the wavelength solutions.
%     \item RMS comparisons between \polsalt\ and \stops\ to quantify differences.
% \end{itemize}

% \todo{Insert table comparing RMS values of wavelength solutions from \polsalt\ and \stops.}

% \todo{Insert figure showing full frame wavelength solution plots.}

% The \texttt{correlate} and \texttt{skylines} methods were tested for accuracy and performance. Key aspects of the testing included:
% \begin{itemize}
%     \item Verifying the accuracy of the cross-correlation function against known standards.
%     \item Ensuring skyline identification was accurate and consistent.
%     \item Performance testing with large data sets to ensure efficiency.
% \end{itemize}

% \todo{Insert graph showing correlation results with known standards.}

% \todo{Insert figure illustrating skyline identification accuracy.}

% MARK: WAV Corr. Checks
\subsection{Cross Correlation Checks} \label{subsec:test_corr}

The \texttt{correlate} method returns plots validating the wavelength solutions and so only has to accept the \polsalt\ \texttt{spectra extraction} method output files as input.

% \begin{itemize}
%     \item Testing correlate functionality using `offset', comparisons of arcs, FSRQ's and BLLac's.
%     \item Any figures showing correlation tests?
% \end{itemize}

% MARK: WAV Sky. Checks
\subsection{Sky Line Checks} \label{subsec:test_sky}

The \texttt{skylines} method returns plots validating the wavelength solutions and so only has to accept either the \iraf\ \texttt{transform} task or \stops\ \texttt{join} method output files as input.

% \begin{itemize}
%     \item Testing skylines using known spectral sky lines.
%     \item Any figures showing skyline tests?
% \end{itemize}

% MARK: Results Unpub.
\section[Application of \textsc{stops}]{Application of \stops} \label{sec:results_unpub}

% \begin{itemize}
%     \item Background information on each object.
%     \item Detailed reductions steps performed on each object.
%     \item Comparison of \polsalt\ results to those obtained using the \stops\ pipeline.
% \end{itemize}

% Application to Spectropolarimetric Standards
% \begin{itemize}
%     \item Science results, what the results can tell us and why it is useful, also comparison of results to FORS1/2 published data, focus on the polarization results.
% \end{itemize}


% Reduction specific tests refer to tests performed to ensure no errant effects are introduced during the reduction process. These tests were designed to validate the accuracy and reliability of the \stops\ pipeline in producing scientifically accurate results.

% % General discussion of sources used as part of testing
% The sources used for testing included a range of spectropolarimetric standards, both highly polarized and non-polarized objects. The tests aimed to check for issues such as data integrity, consistency of wavelength calibration, and accuracy of polarization measurements.

% \todo{
%     \begin{itemize}
%         \item General discussion of testing (I.E. not this test was done specifically this source, more along the lines of these tests were done to check this issue, seen here using this source for example.)
%     \end{itemize}
% }

% % MARK: Polarization tests
\subsection{Polarization Parameters} \label{subsec:test_polarization}

% Polarization parameters validation from known sources.
%     \begin{itemize}
%         \item Polarization tested using 3C 279, 4C+01.02, and preliminary testing data provided by David.
%         \item Polarization tested using spectropolarimetric standards (4 highly polarized, 2 non-polarized).
%         \item Tabulate sources used, with their properties.
%         \item Figures showing comparison plots of polarization parameters from \polsalt\ and \stops.
%     \end{itemize}

% The accuracy of polarization parameters was tested using several known sources, including:
% \begin{itemize}
%     \item 3C 279
%     \item 4C+01.02
%     \item Data provided by David (preliminary testing data, not included in publications but used during pipeline development)
% \end{itemize}

% % MARK: Spectropolarimetric Standards
\subsection{Spectropolarimetric Standards}

% Testing included the use of spectropolarimetric standards, comprising four highly polarized and two non-polarized objects. The specific tests performed were:
% \begin{itemize}
%     \item Background information on each object.
%     \item Detailed reductions steps performed on each object.
%     \item Comparison of \polsalt\ results to those obtained using the \stops\ pipeline.
% \end{itemize}

% \todo{Insert table listing the spectropolarimetric standards used, with their properties.}

% \todo{Insert figure showing comparison plots of polarization parameters from \polsalt\ and \stops.}

% \todo{Spectropolarimetric standards (4 highly polarized, 2 non-polarized)
%     \begin{itemize}
%         \item (Same as ch04 with science results)
%         \item Background on objects
%         \item Reductions
%         \item Actual results - comparison of polsalt results to supplementary pipeline results
%         \item Science results, what the results can tell us and why it is useful, also comparison of results to FORS1/2 published data, focus on the polarization results
%     \end{itemize}
% }

% MARK: Results Pub.
\section{Application in Publications} \label{sec:results_pub}

% Application in Publications
% \begin{itemize}
%     \item Summarize the results of the publications appended to appendix.
% \end{itemize}

% \todo{Summary of results from papers in appendix.
%     \begin{itemize}
%         \item Hester paper(s)
%         \item Joleen proceedings and work
%         \item My proceedings
%     \end{itemize}
% }

% \todo{3C 279 and 4C+01.02
%     \begin{itemize}
%         \item Give Background on objects, Reduction steps, and Science results (what the results can tell us and why it is useful)
%         \item (comparison of polsalt results to supplementary pipeline results will be in testing)
%     \end{itemize}
% }
