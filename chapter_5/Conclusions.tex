\chapter{Conclusions} \label{ch:05}

% • Background and motivation
% - What are you talking about?
% - What questions are you trying to answer?
% - What’s the bigger picture?
% • What did you do?
% - What did you observe?
% - Experimental and analysis method(s)
% - What theory/models are you using?
% • What did you find?
% • What does it mean? What do you conclude?

\todo{A summary of the dissertation, main focus on the results and that the supplementary pipeline is a success since it allows an alternate method using IRAF to wavelength calibrate the polsalt data.}

\section{Future Work}

It is possible to complete the data reductions of \gls{SALT}/\gls{RSS} spectropolarimetric data using only the \polsalt\ pipeline.
This does not negate the fact that better tools and software better allow us to focus on the results of observations rather than the data reduction process.
Due to the limitations inherent in software designed for use strictly as a pipeline, there is a lack of flexibility in data reduction processes, an over-reliance on the software to provide accurate results, and a lack of urgency when keeping `completed' software up-to-date.
% Point in case, \polsalt\ is written in Python~$2$, which was deprecated in $2020$, while \iraf\ is referred to as legacy software, receiving only community driven support.

In this regard, the development of the \stops\ software is a step in the right direction.
Although \stops\ was developed as a workaround to allow for the integration of alternate wavelength calibrations with \polsalt, it utilizes a well-supported programming language, Python~$3$, as well as actively developed \gls{PyPI} packages, and is compatible with all devices capable of running Python.
That said, the software is not without its limitations.

Future work would involve the continued development and maintenance of the \stops\ software, further integration of the software with \polsalt, and possibly the development of a user-friendly interface or modification of the \polsalt\ \gls{GUI} to include the \stops\ functionality.
Wavelength calibrations produced using Python are already supported, but further integration of non-standard wavelength solutions would also be beneficial, as the current wavelength solutions are limited to Chebyshev and Legendre polynomials.
Automation of the wavelength calibration process may also be considered, but development of this functionality depends heavily on the development of the \polsalt\ pipeline.
