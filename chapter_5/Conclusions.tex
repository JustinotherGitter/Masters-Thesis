\chapter{Conclusions} \label{ch:05}

This chapter contains an overview of the \stops\ software, with a focus on its development and testing (\autoref{sec:stops_summary}), the findings from the application of \stops\ to \gls{SALT}/\gls{RSS} spectro\-polarimetric data (\autoref{sec:findings}), and a discussion on the future of the \stops\ software (\autoref{sec:future}).

% Summary of \stops\ development and testing
\section[A Broad Overview of \textsc{stops}]{A Broad Overview of \stops} \label{sec:stops_summary}

% What was the goal of \stops?
Means to provide a more streamlined wavelength calibration process for \gls{SALT}/\gls{RSS} spectropolarimetric data were desired, specifically when working with \gls{SALT} arc~lamps with poor uniform spectral coverage, e.g., the \gls{Ar} arc lamp used in conjunction with the PG$0300$ grating.
% Why was \stops developed?
This led to the development of the \stops\ software, and, more specifically, the \stops\ \texttt{split} and \texttt{join} methods (see \autoref{subsec:stops_split}, and \ref{subsec:stops_join}), which allow alternate wavelength calibrations to replace the \polsalt\ \texttt{wavelength calibration}.

With the introduction of alternate wavelength solutions, the \stops\ software was further developed to allow the user to quickly identify any faults in the wavelength solutions.
This was achieved through the development of the \stops\ \texttt{skylines} and \texttt{correlate} methods (see \autoref{subsec:stops_skyline}, and \ref{subsec:stops_correlate}), which allow for the identification of offsets in skyline features and the comparison of the wavelength solutions across $O$- and $E$-beams, respectively.

% What were the results of the reduction-specific testing?
To ensure the accuracy and reproducibility of the \stops\ software, testing was conducted, for each method, on both synthetic and real-world data (see \autoref{sec:test_stops}).
A comparison of the \texttt{split} method's inputs and outputs showed that the data were unaffected during conversion from a dual beam to a single beam format, allowing accurate wavelength solutions to be found.
The largest discrepancies introduced by the \texttt{split} method, other than the splitting of the $O$- and $E$-beams to separate files, are the cropping of the non-exposed regions of the \gls{CCD}, and the removal of extensions, both of which are not used during \iraf\ wavelength calibrations.

Similarly, when comparing the output of the \stops\ \texttt{join} and \polsalt\ \texttt{wavelength calibration} methods the alternate wavelength solutions are unaffected, although they differ slightly due to the differing wavelength calibration methods, when converted back to the \polsalt\ format, thus providing \polsalt\ with accurate alternate wavelength solutions for both polarimetric beams.
The largest discrepancies introduced by the \texttt{join} method are the non-wavelength related processes as handled by \stops\ to substitute the \polsalt\ \texttt{wavelength calibration}, specifically the handling of Wollaston corrections near the beam intersection (see \autoref{par:wollaston}) and the \gls{CRR} being applied to the \gls{SCI} extension instead of the \gls{BPM} extension.
Both of these discrepancies, however, do not affect \polsalt, the rest of the reduction process, or the final results.

% What were the results of the check-specific testing?
It is possible, and recommended, to ensure the accuracy of the wavelength solutions within whichever alternate wavelength calibration software the user decides to use, e.g., \iraf, or Python. The \stops\ \texttt{skylines} and \texttt{correlate} methods were, therefore, developed to determining the accuracy of the wavelength solutions.
To eliminate variability inherent in real-world data, synthetic data were generated and used to test these methods.
By introducing known offsets to synthetic data, the recovery of these offsets through the \stops\ check-specific methods ensured that the methods were functioning as intended.

The \texttt{skylines} method is able to recover offsets in generated skyline features within a symmetrical $\approx4$~\AA\ region, limited by the nearest-neighbors of the relevant identifiable skyline, with an accuracy of $\approx1$~\AA\ due to the limitations of the spectral resolution.
The \texttt{correlate} method is able to recover offsets introduced to wavelength calibrated spectra, for each \gls{CCD}, and for comparisons of the $O$- and $E$-beams within a single file, or for the same polarimetric beam across multiple files, with an accuracy of $\approx1$~\AA, limited once again by the spectral resolution.

% Findings
\section{Scientific results using STOPS} \label{sec:findings}

A number of publications have been produced during the development of \stops.
Most notably are the results for the papers discussed in \autoref{sec:app_stops}, which present the findings for a selection of sources, i.e., \citet{Buckley191221B, Cooper_HEASA2021, Cooper_HEASA2022, Schutte4C0102}.
Additionally, the \stops\ software has been used extensively in \citet{Barnard_HEASA2021, Barnard_2024, Barnard_thesis}.

% Summary of contributions (without \stops)
% \citet{Buckley191221B} presented the results \dots

% Summary of contributions (with \stops)
\citet{Cooper_HEASA2021}, and later \citet{Cooper_HEASA2022}, present results for the blazar $3$C~$279$, reduced using alternate \iraf\ wavelength calibrations.
It was shown that the \stops\ software has no noticeable effect on the spectral and polarization properties, and that the \iraf\ wavelength solutions were accurate and reliable.
The blazar was observed during periods of enhanced activity as well as during a period of flaring $\gamma$-ray activity on 31 March 2017.
The normalized relative flux spectra of $3$C~$279$ showed a consistent spectral shape across the various epochs, only deviating slightly during the epoch of flaring, as noted by the increase at shorter wavelengths in the $\sim 6000 - 9000$~\AA\ wavelength range.
The linear polarization remained consistent across the wavelength range at each epoch but varied across epochs, with a minimum linear polarization occurring during the period of flaring.
The angle of polarization was also consistent across the wavelength range while varying across epochs.

Both the spectral and linear polarization results seem counter-intuitive for a spectropolarimetric observation of a flaring blazar.
An increase in the blue region of the spectrum is generally associated with the thermal, non-polarized emission arising in the accretion disk component during periods of low non-thermal, polarized, synchrotron emission.
A state of low, polarized, synchrotron emission, often related to periods of quiescence, is further corroborated by the minimum occurring in the linear polarization during this epoch.
Past measurements of the polarization of 3C~$279$ place the polarization during quiescence at $\Pi \approx 10 \%$ \citep[see e.g., ][]{3C279_xray}, falling within the observed $\Pi = 9.2 \pm 1.1\%$.
Furthermore, 3C~$279$ is known to show large, rapid minute-scale, variations in the polarization angle \citep{3C279_var}, possibly linked to repeating $\gamma$-ray activity \citep{3C279_repeat}.
The disparity between the flaring $\gamma$-ray and apparent quiescent optical regimes may be due to for example, a turbulent magnetic field located in the optical emission region.
An unordered magnetic field would lead to depolarization effects, and aligns with the polarization results in \autoref{fig:3C_279_specpol}.

\citet{Cooper_HEASA2022} further reported on results for the spectrophotometric standard star, Hiltner~$652$, reduced with alternative \iraf\ wavelength calibrations applied.
The reduced $Q$ and $U$ Stokes parameters of Hiltner~$652$ showed no systematic deviation from previously published results, indicating that the wavelength calibration process did not introduce any errors into the polarization calculations.

\citet{Schutte4C0102} presented results for the blazar $4$C~+$01.02$, reduced using the \stops\ software.
The spectropolarimetric results were used, alongside multi-wavelength observations, to model the \gls{SED} of the blazar across periods of quiescent and flaring activity.
Using the known redshift for $4$C~+$01.02$ of $z = 2.1$, emission features in the normalized, continuum subtracted, spectra were identified.
These features are more prominent during the quiescent state due to the fainter non-thermal continuum emission as compared to the flaring state.
Optical spectropolarimetry of $4$C~$01.02$ showed a consistent polarization percentage across the observed wavelength range, averaging $10.04 \pm 1.03 \%$ and $1.72 \pm 0.93 \%$, for the flaring and quiescent state, respectively.

% Did \stops\ meet its objectives?
It has been shown that the \stops\ software allows for the integration of alternate wavelength calibrations with \polsalt, and that the software is accurate and reliable, meeting the objectives laid out at the offset of its development.

% Future work
\section{Future Development} \label{sec:future}

It is possible to complete the data reductions of \gls{SALT}/\gls{RSS} spectropolarimetric data using only the \polsalt\ pipeline.
This does not negate the fact that better tools and software better allow us to focus on the results of observations rather than the data reduction process.
Due to the limitations inherent in software designed for use strictly as a pipeline, there is a lack of flexibility in data reduction processes, an over-reliance on the software to provide accurate results, and a lack of urgency when keeping `completed' software up-to-date.
% Point in case, \polsalt\ is written in Python~$2$, which was deprecated in $2020$, while \iraf\ is referred to as legacy software, receiving only community driven support.

In this regard, the development of the \stops\ software is a step in the right direction.
Although \stops\ was developed as a workaround to allow for the integration of alternate wavelength calibrations with \polsalt, it utilizes a well-supported programming language, Python~$3$, as well as actively developed \gls{PyPI} packages, and is compatible with all devices capable of running Python.
That said, the software is not without its limitations.

Future work would involve the continued development and maintenance of the \stops\ software, further integration of the software with \polsalt, and possibly the development of a user-friendly interface or modification of the \polsalt\ \gls{GUI} to include the \stops\ functionality.
Wavelength calibrations produced using Python are already supported, but further integration of non-standard wavelength solutions would also be beneficial, as the current wavelength solutions are limited to Chebyshev and Legendre polynomials.
Automation of the wavelength calibration process may also be considered, but development of this functionality depends heavily on the development of the \polsalt\ pipeline.
