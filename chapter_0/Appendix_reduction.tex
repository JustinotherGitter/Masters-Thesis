% MARK: Appendix - Reduction
\chapter{The Modified Reduction Process} \label{app:reduction}

% MARK: listings variables
\newcommand{\obj}{<\textit{OBJ}>}
\newcommand{\obsdate}{<\textit{OBSDATE}>}

\newcommand{\oarc}{<\textit{O-beam~ARC}>}
\newcommand{\earc}{<\textit{E-beam~ARC}>}
\newcommand{\ofiles}{<\textit{O-beam~FILES}>}
\newcommand{\efiles}{<\textit{E-beam~FILES}>}
\newcommand{\file}{<\textit{FILE(S)}>}

\newcommand{\id}{01\_identify.cl}
\newcommand{\reid}{02\_reidentify.cl}
\newcommand{\fit}{03\_fitcoords.cl}
\newcommand{\tran}{04\_transform.cl}

This section of the Appendix aims to provide a minimum working example of the commands necessary to reduce \polsalt\ data using \stops\ and \iraf. It contains the commands necessary to activate all software and run through the reduction process but makes no attempt at discussion.

Both \polsalt\ and \iraf\ are launched from the default \gls{CLI} but use independent interfaces during the reduction process. To distinguish which window is in focus, the `\texttt{\$}' token is used for default \gls{CLI} commands while the `\texttt{cl>}' and `\texttt{>\->\->}' tokens are used for \iraf's `xgterm' single- and multi-line commands, respectively. \gls{CLI} commands may also require a specific Python environment. The environment being used is denoted using the `\texttt{(salt)\$}' or `\texttt{(base)\$}' tokens for the `salt' or default `base' environments.%
\footnote{It is assumed a Python environment manager, such as `conda', is installed.}

General instructions for the reduction process (i.e., not \gls{CLI} line-fed commands) may either be discussed outside a `Listing' environment or included as part of a `Listing' environment preceded with a comment `\texttt{\#}' token. Finally, \polsalt\ implements a \gls{GUI} and thus takes no line-fed commands. As such, instructions referring to the \polsalt\ \gls{GUI} also follow those of the general instructions.

As a final note, some parameters are distinguished using `<angle brackets>'. They signify necessary parameters that may vary from reduction to reduction. Notable uses of this notation include the date of observation, \obsdate\ (formatted `YYYYMMDD'), the split science \gls{FITS} files, \ofiles\ or \efiles, the split arc \gls{FITS} files, \oarc\ or \earc, and a wildcard symbol, <\texttt{*}>.

% MARK: Reduction Procedure
\vspace{0.5\baselineskip}\hrule

\begin{figure}[h!]
    \centering
    \begin{minipage}{8cm}
        \dirtree{%
            .1 \obsdate .
            .2 raw .
            .3 P\obsdate<*>.fits .
            .2 sci .
            .3 mxgbpP\obsdate<*>.fits .
        }
    \end{minipage}
    \caption{The typical minimal file structure of data provided by \gls{SALT}.}
    \label{dir:pre_red}
\end{figure}

Ensure the data is formatted in a file structure similar to that in \autoref{dir:pre_red}. Data located in the `sci' folder is often provided by \gls{SALT} but is not strictly necessary to begin the reduction process. If `mxgbp' prefixed data is available, the reductions may be begun starting at \autoref{code:stops_split}. The \polsalt\ \gls{GUI} is launched from the default \gls{CLI} running the commands in \autoref{code:polsalt_launch}.

\begin{lstlisting}[
    language=bash,
    caption={Launching the \polsalt\ \gls{GUI}},
    label=code:polsalt_launch]
$ cd ~/polsalt-beta
$ conda activate salt
(salt)$ python -W ignore reducepoldataGUI.py &
\end{lstlisting}

Refer to \autoref{fig:polsalt_gui} for a depiction of the \polsalt\ \gls{GUI}. To complete the \polsalt\ pre-calibrations, and with the \gls{GUI} in focus:
\begin{itemize}
    \item Ensure that the `POLSALT code directory' is correct, i.e., `\mytilde/polsalt-beta'.
    \item Set the `Top level data directory' to \obsdate.
    \item Ensure `Raw data directory' is correct.
    \item Ensure `Science data directory' is correct.
    \item Select `Raw image reduction' from the `Data reduction step' drop down menu.
    \item Check the tick boxes of all raw images to be processed (include the arc) in the display box covering the lower half of the \gls{GUI}.
    \item Proceed with the reductions by clicking the `OK' button.
\end{itemize}

The pre-calibrated `mxgbp' \gls{FITS} files are now available in the `sci' folder. The files may be split using \stops\ by running the commands in \autoref{code:stops_split}.

\begin{lstlisting}[
    language=bash,
    caption={Splitting data using \stops},
    label=code:stops_split]
$ cd (*@\obsdate@*)/sci
$ conda activate
(base)$ python -m STOPS . split
\end{lstlisting}

The split $O$- and $E$-beam \gls{FITS} files are now available. The \iraf\ wavelength calibrations are now run. The \iraf\ xgterm \gls{CLI} is launched using \autoref{code:iraf_launch}.

\begin{lstlisting}[
    language=bash,
    caption={Launching \iraf\ in xgterm},
    label=code:iraf_launch]
$ cd ~/iraf
$ xgterm -sb &
cl> conda activate salt
cl> cl
cl> cd (*@\obsdate@*)/sci
cl> noao
cl> twodspec
cl> longslit
cl> unlearn longslit
cl> longslit.dispaxis=1
\end{lstlisting}

\pagebreak

The \iraf\ \texttt{identify} task requires an average feature width, `fwidth', as a parameter. The width of a feature may be found in \iraf\ using the \texttt{implot} task%
\footnote{See \protect\url{https://astro.uni-bonn.de/~sysstw/lfa_html/iraf/plot.implot.html} for documentation on the \texttt{implot} task.}
along with cursor commands, but may also be found using \gls{FITS} viewing software, such as \texttt{ds9}.%
\footnote{See \protect\url{https://sites.google.com/cfa.harvard.edu/saoimageds9} for documentation on the \texttt{ds9} software.}
The \texttt{identify} task may be run using the commands in \autoref{code:iraf_id}.

\begin{lstlisting}[
    language=bash,
    caption={Running the \iraf\ \texttt{identify} task},
    label=code:iraf_id]
cl> mkscript (*@\id@*)
cl> # Add identify to (*@\id@*) twice, for both beams
cl> # Edit the parameters of (*@\id@*) in a text editor
cl> # Paste an identify script into the CLI, resulting in:
cl> 
cl> identify ("(*@\oarc@*)",
>>> "", "", section="middle line", database="database", coordlist="linelists$idhenear.dat", units="", nsum="10", match=-3., maxfeatures=50, zwidth=100., ftype="emission", fwidth=8., cradius=5., threshold=0., minsep=2., function="spline3", order=2, sample="*", niterate=0, low_reject=3., high_reject=3., grow=0., autowrite=no, graphics="stdgraph", cursor="", aidpars="")
\end{lstlisting}
{\parskip=0pt The} \texttt{identify} task will launch an interactive window. Cursor commands refer to keys that provide unique functionality to the interactive \iraf\ tasks. The cursor commands for \texttt{identify} allow the arc lines to be identified using `m' (and typing the relevant wavelength), while `d' and `i' will delete a single and all identified arc lines, respectively. The `f' cursor command will perform a preliminary fit which can be quit using the `q' cursor command. The `l' cursor command will attempt to identify any unidentified arc lines. Once complete, a figure of the identified lines may be saved using `:labels coord' and `:.snap eps', and the task safely quit with the `q' cursor command.%
\footnote{See \protect\url{https://astro.uni-bonn.de/~sysstw/lfa_html/iraf/noao.onedspec.identify.html} for documentation on the \texttt{identify} task.}
The \texttt{identify} procedure is repeated, replacing \oarc\ with \earc.

The \texttt{reidentify} task may be run using the commands in \autoref{code:iraf_reid}.

\begin{lstlisting}[
    language=bash,
    caption={Running the \iraf\ \texttt{reidentify} task},
    label=code:iraf_reid]
cl> mkscript (*@\reid@*)
cl> # Add reidentify to (*@\reid@*) twice, for both beams
cl> # Edit the parameters of (*@\reid@*) in a text editor
cl> # Paste a reidentify script into the CLI, resulting in:
cl> 
cl> reidentify ("(*@\oarc@*)",
>>> "(*@\oarc@*)", "yes", "", "", interactive="no", section="middle line", newaps=yes, override=no, refit=yes, trace=yes, step="10", nsum="10", shift="0.", search=0., nlost=0, cradius=5., threshold=0., addfeatures=no, coordlist="linelists$idhenear.dat", match=-3., maxfeatures=50, minsep=2., database="database", logfiles="logfile", plotfile="", verbose=yes, graphics="stdgraph", cursor="", aidpars="")
\end{lstlisting}

\pagebreak

\noindent The \texttt{reidentify} task will run autonomously so long as the \texttt{interactive} parameter is set to ``no''.%
\footnote{See \protect\url{https://astro.uni-bonn.de/~sysstw/lfa_html/iraf/noao.onedspec.reidentify.html} for documentation on the \texttt{reidentify} task.}
Repeat the \texttt{reidentify} procedure, replacing \oarc\ with \earc\ at both the `reference' and `image' parameter locations.

The \texttt{fitcoords} task may be run using the commands in \autoref{code:iraf_fitcoords}.

\begin{lstlisting}[
    language=bash,
    caption={Running the \iraf\ \texttt{fitcoords} task},
    label=code:iraf_fitcoords]
cl> mkscript (*@\fit@*)
cl> # Add fitcoords to (*@\fit@*) twice, for both beams
cl> # Edit the parameters of (*@\fit@*) in a text editor
cl> # Paste a fitcoords script into the CLI, resulting in:
cl> 
cl> fitcoords ("(*@\oarc\ (exclude the file extension)@*)",
>>> fitname="", interactive=yes, combine=no, database="database", deletions="deletions.db", function="chebyshev", xorder=6, yorder=6, logfiles="STDOUT,logfile", plotfile="plotfile", graphics="stdgraph", cursor="")
\end{lstlisting}
{\parskip=0pt The} \texttt{fitcoords} task will launch an interactive window. The x- and y-axis being plotted may be changed using the `x' or `y' cursor commands followed by the desired data axis (`x' for the x-axis, `y' for the y-axis, or `r' for the residuals).%
\footnote{See \protect\url{https://astro.uni-bonn.de/~sysstw/lfa_html/iraf/noao.twodspec.longslit.fitcoords.html} for documentation on the \texttt{fitcoords} task.}
Repeat the \texttt{fitcoords} procedure, replacing \oarc\ with \earc.

The \texttt{transform} task may be run using the commands in \autoref{code:iraf_transform}.

\begin{lstlisting}[
    language=bash,
    caption={Running the \iraf\ \texttt{transform} task},
    label=code:iraf_transform]
cl> mkscript (*@\tran@*)
cl> # Add transform to (*@\tran@*) twice, for both beams
cl> # Edit the parameters of (*@\tran@*) in a text editor
cl> # Paste a transform script into the CLI, resulting in:
cl> 
cl> transform ("(*@@\ofiles@*)",
>>> "(*@t//@\ofiles@*)", "(*@\oarc@*) (exclude the file extension)", minput="", moutput="", database="database", interptype="linear", x1="INDEF", x2="INDEF", dx="INDEF", nx="INDEF", xlog="no", y1="INDEF", y2="INDEF", dy="INDEF", ny="INDEF", ylog="no", flux="yes", blank="INDEF", logfiles="STDOUT,logfile")
\end{lstlisting}
{\parskip=0pt The} \texttt{transform} task will run autonomously.%
\footnote{See \protect\url{https://astro.uni-bonn.de/~sysstw/lfa_html/iraf/noao.twodspec.longslit.transform.html} for documentation on the \texttt{transform} task.}
Repeat the \texttt{transform} procedure, replacing the \ofiles\ and \oarc\ with \efiles\ and \earc\ at both parameter locations. Inspect the transformed images, most notably the arc images, using any \gls{FITS} viewer as a cursory check that the wavelength calibrations were completed without error.


The `gain' and `read noise' are now needed as the cosmic-ray rejection of the \stops\ \texttt{join} method accepts them as parameters. These parameters may be found using the `\textit{GAINSET}' and `\textit{ROSPEED}' keywords in the \gls{FITS} headers. The cosmic ray rejection defaults to \textit{GAINSET}$ = $`FAINT', and \textit{ROSPEED}$ = $`SLOW'. If the gain and read noise values differ from the defaults, the parameters should be updated when running \texttt{join}.%
\footnote{The read noise and gain may be determined from \citet{SALT_CFP}, specifically \href{https://pysalt.salt.ac.za/proposal\_calls/current/ProposalCall.html\#t.4}{Table 6.1} and \href{https://pysalt.salt.ac.za/proposal\_calls/current/ProposalCall.html\#t.5}{Table 6.2}.}

The \stops\ \texttt{join} method may be run using the commands in \autoref{code:stops_join}.

\begin{lstlisting}[
    language=bash,
    caption={Joining the data using \stops},
    label=code:stops_join]
$ cd (*@\obsdate@*)/sci
$ conda activate
(base)$ python -m STOPS . join
\end{lstlisting}

The \stops\ \texttt{skylines} method may be run on any `joined' or transformed \gls{FITS} files, \file, using the commands in \autoref{code:stops_sky}.

\begin{lstlisting}[
    language=bash,
    caption={Running the \stops\ \texttt{skylines} method},
    label=code:stops_sky]
$ cd (*@\obsdate@*)/sci
$ conda activate
(base)$ python -m STOPS . skylines (*@\file@*)
\end{lstlisting}

The `\texttt{sci/}' directory may now be optionally organized by running the commands in \autoref{code:gen_clean}, moving all the files relevant to the wavelength calibrations into either the `\texttt{database}' or `\texttt{split\_files}' directories.

\begin{lstlisting}[
    language=bash,
    caption={Directory cleanup for \polsalt},
    label=code:gen_clean]
$ cd (*@\obsdate@*)/sci
$ mkdir split_files
$ mv *beamO* *beamE* *arcO* *arcE* split_files/
$ mv *.eps *.cl *.db database/
\end{lstlisting}

The \polsalt\ \texttt{spectra extraction} is now run. If the \polsalt\ \gls{GUI} was closed it should now be reopened using \autoref{code:polsalt_launch}. With the \gls{GUI} in focus:
\begin{itemize}
    \item Ensure all directories are still correct.
    \item Select `Spectra extraction' from the `Data reduction step' drop down menu.
    \item Check the tick boxes of all wavelength calibrated images to be processed (exclude the arc) in the display box covering the lower half of the \gls{GUI}.
    \item Proceed with the reductions by clicking `OK'.
\end{itemize}

The \polsalt\ \texttt{spectra extraction} is interactive and will launch a separate \gls{GUI} for the background subtraction and spectral extraction (see \autoref{fig:polsalt_gui_spec}). The background and spectral regions to be extracted may be adjusted, noting that adjustments affect both $O$- and $E$-beams. Once both background regions contain no trace and the spectral region fully contains only the science trace, the reduction may be completed by clicking `OK'.

\pagebreak

The \stops\ \texttt{correlate} method may now be run on any `joined' \gls{FITS} files by running the commands in \autoref{code:stops_corr}.

\begin{lstlisting}[
    language=bash,
    caption={Running the \stops\ \texttt{correlate} method},
    label=code:stops_corr]
$ cd (*@\obsdate@*)/sci
$ conda activate
(base)$ python -m STOPS . correlate (*@\file@*)
\end{lstlisting}

The \polsalt\ \texttt{raw Stokes calculation}, \texttt{final Stokes calculation}, and \texttt{results visualisation} may now be completed. For the last time, if the \polsalt\ \gls{GUI} was closed it should now be reopened using \autoref{code:polsalt_launch}. With the \gls{GUI} in focus:
\begin{itemize}
    \item Ensure all directories are still correct.
    \item Select `Raw Stokes calculation' from the `Data reduction step' drop down menu.
    \item Check the tick boxes of all the extracted spectra images to be processed in the display box covering the lower half of the \gls{GUI}.
    \item Proceed with the \texttt{raw Stokes calculation} by clicking `OK'.
    \item Select `Final Stokes calculation' from the `Data reduction step' drop down menu.
    \item Check the tick boxes of all the ``raw Stokes'' images to be processed in the display box covering the lower half of the \gls{GUI}.
    \item Proceed with the \texttt{Final Stokes calculation} by clicking `OK'.
    \item Select `Results visualisation - interactive' from the `Data reduction step' drop down menu.
    \item Check the tick boxes of the ``final Stokes'' image to be visualized in the display box covering the lower half of the \gls{GUI}.
    \item Proceed with the \texttt{visualisation} by clicking `OK'.
\end{itemize}

The \polsalt\ \texttt{visualisation} is interactive and will launch a separate \gls{GUI} (See \autoref{fig:polsalt_gui_vis}). The \gls{GUI} may be used to change the binning and parameters of the plot before saving the plot to a PDF file.

This concludes the minimum working example of the \polsalt\ reduction process when substituting the \polsalt\ \texttt{wavelength calibrations} with those done in \iraf. Aside from the final results, the file structure after reductions should resemble something akin to that provided in \autoref{dir:post_red}.

\begin{figure}
    \centering
    \begin{minipage}{9cm}
        \dirtree{%
            .1 \obsdate .
            .2 raw .
            .3 P\obsdate<*>.fits .
            .2 sci .
            .3 database .
            .4 \id .
            .4 \reid .
            .4 \fit .
            .4 \tran .
            .4 deletions.db .
            .4 fcarcE00<\#\#> .
            .4 fcarcO00<\#\#> .
            .4 idarcE00<\#\#> .
            .4 idarcO00<\#\#> .
            .4 <*>.eps .
            .3 split\_files .
            .4 arcE00<\#\#>.fits .
            .4 arcO00<\#\#>.fits .
            .4 beamE<*>.fits .
            .4 beamO<*>.fits .
            .4 tarcE00<\#\#>.fits .
            .4 tarcO00<\#\#>.fits .
            .4 tbeamE<*>.fits .
            .4 tbeamO<*>.fits .
            .3 \obsdate\_geom.txt .
            .3 \obsdate\_filtered.txt .
            .3 cwmxgbpP\obsdate<*>.fits .
            .3 ecwmxgbpP\obsdate<*>.fits .
            .3 mxgbpP\obsdate<*>.fits .
            .3 wmxgbpP\obsdate<*>.fits .
            .3 <*>.log .
            .3 \obj\_c0\_h<*>\_01.fits .
            .3 \obj\_c0\_1\_stokes.fits .
            .3 \obj\_c0\_1\_stokes\_<BIN>\_Ipt.txt .
            .3 \obj\_c0\_1\_stokes\_<BIN>\_Ipt.pdf .
        }
    \end{minipage}
    \caption{The typical file structure after completing the reduction process.}
    \label{dir:post_red}
\end{figure}
