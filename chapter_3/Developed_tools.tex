\chapter{Developed Tools}

This chapter contains an overview of why supplementary tools were deemed necessary for an already complete reduction process(\S~\ref{sec:polsalt_limits}), which aspects of the reduction process have been altered, replaced, or added (\S~\ref{sec:mod_tools}, \ref{sec:add_tools}), and finally what an updated reduction process consists of using a combination of all software.

% \todo{
%   \begin{itemize}
%     \item Mention difficulty with PG0300?
%     \item Mention GUI difficulties?
%   \end{itemize}
% }

\section{Limitations of POLSALT and the Need for a Supplementary Tool} \label{sec:polsalt_limits} % Rename

% Why
The creation of supplementary tools for \textsc{polsalt} spectropolarimetric reductions stem from, primarily, the limitations of the wavelength calibration phase and a need for a way to compare wavelength solutions across matching $O$ and $E$ polarization beams. Due to the time-consuming process of recalibrating the wavelength solutions it is not feasible to perform the wavelength calibrations time and time again for any amount of reductions larger than a handful of observations.
\prgph

% How
One solution is to use a well established tool to perform the wavelength calibration, one which allows for rapid recalibrations as well as provides a familiar interface with which the user can analyze their wavelength solution. \textsc{iraf} provides this familiar environment and reliability, even considering it's age and lack of further development. Unfortunately, \textsc{iraf} is unable to parse the files as is and a need for formatting is necessary for integration purposes. This restructuring works both ways as once the \textsc{iraf} reductions are complete the format must be reformatted to match that of the \textsc{polsalt} output such that reductions can carry on in \textsc{polsalt}.


\todo{Can mention other faults very briefly, but the main takeaway should be the need for a way to do the wavelength calibration with more control as well as the need to check the O/E beam wavelength calibrations against one another (since very accurate wavelength calibration necessary for stokes parameter calculations)}


\section{Wavelength calibrations using the Supplementary Pipeline and IRAF} \label{sec:mod_tools}

% \todo{short description of what is to be discussed}

\subsection{Splitting the uncalibrated wavelength files}

% \todo{All processes run in pipeline split
%   Focus on \textbf{why}. Parsing polsalt mxgbp frame into something useable by IRAF and making sure the header reflects the changes}

\subsection{IRAF wavelength calibration}\label{subsec:IRAF_wav_cal}

% \todo{All wavelength calibration steps - Again, focus on \textbf{why} instead of how
%   \begin{itemize}
%     \item Identify
%     \item Reidentify
%     \item Fitcoords
%   \end{itemize}
% }

% \todo{(Optional) Transform (mention good for sanity checks which is not possible using the pure polsalt implementation)}

\subsection{Joining the wavelength calibrated files}

% \todo{All processes run in pipeline join.
%   Focus on \textbf{why}. Parsing \textsc{iraf} frames to be used by POLSALT and making sure the header and extensions reflect the changes}

\section{Additional Tools}\label{sec:add_tools}

\subsection{Cross correlation}

% \todo{Why a cross correlation necessary and how it works}

\subsection{Skyline comparisons}

% \todo{Again, why a skyline comparison necessary and how it works. Also, how the frame is transformed (\textsc{iraf} bypassed) and that the flux is not conserved so only for checking and not for science use.}

\section{General Reduction Procedure}

% \todo{General reduction procedure from raw data to finalized results
%   \begin{itemize}
%     \item This includes \textsc{polsalt} pre-reductions, splitting, \textsc{iraf} wavelength calibrations, checking, joining, and \textsc{polsalt} finalization. (Include Relative flux calibrations for `shape correcting' spectrum??)
%   \end{itemize}
% }