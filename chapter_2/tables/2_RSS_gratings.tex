\begin{table}[t]

  \centering

  \caption{Gratings available for use with the \gls{RSS}. Table adapted from the \gls{SALT} call for proposals (2023).}
  \label{table:RSS_gratings}

  \begin{tabular}{lcccc}
    \toprule
    Grating Name &
    \begin{tabular}[c]{@{}c@{}}Wavelength\\Coverage (\AA)\end{tabular} &
    \begin{tabular}[c]{@{}c@{}}Usable Angles\\(\degree)\end{tabular} &
    \begin{tabular}[c]{@{}c@{}}Bandpass per tilt\\(\AA)\end{tabular} &
    \begin{tabular}[c]{@{}c@{}}Resolving\\Power ($1.25\arcsec$ slit)\end{tabular} \\ \midrule
    PG0300\footnotemark{} & $3700 - 9000$ & $ $ & $3900/4400$ & $250 - 600$ \\
    PG0700\footnotemark[\value{footnote}] & $3200 - 9000$ & $3.0 - 7.5$ & $4000 - 3200$ & $400 - 1200$ \\
    PG0900 & $3200 - 9000$ & $12 - 20$ & $\sim3000$ & $600 - 2000$ \\
    PG1300 & $3900 - 9000$ & $19 - 32$ & $\sim2000$ & $1000 - 3200$ \\
    PG1800 & $4500 - 9000$ & $28.5 - 50$ & $1500 - 1000$ & $2000 - 5500$ \\
    PG2300 & $3800 - 7000$ & $30.5 - 50$ & $1000 - 800$ & $2200 - 5500$ \\
    PG3000 & $3200 - 5400$ & $32 - 50$ & $800 - 600$ & $2200 - 5500$ \\
    \bottomrule
  \end{tabular}

\end{table}
\footnotetext{The PG0300 surface relief grating has been replaced with the PG0700 \gls{VPH} grating as of November 2022 but has been included here as observations using the PG0300 are used in later sections.}
